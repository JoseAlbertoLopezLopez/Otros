% !TEX encoding = UTF-8 Unicode
\chapter{Conclusiones}

A lo largo del desarrollo del presente trabajo se propuso detectar automáticamente, por medio de un algoritmo, tres tipos de movimientos (considerados anómalos) presentes en la conducción de automóviles.
Dicho algoritmo fue diseñado para hacer uso de la teoría del análisis de Fourier, así como de diversas mediciones de la dinámica de un vehículo. 
Esto con el objetivo de que, a través de la información obtenida por medio del análisis de los patrones relativos a la dinámica del vehículo, se pudiesen detectar los tres eventos de interés.
Con los resultados presentados hasta ahora se demuestra que existe un grado de eficiencia aceptable en la detección de los movimientos de conducción analizados en este trabajo.

Se considera que las contribuciones más importantes, logradas del desarrollo del presente trabajo de tesis son las siguientes:

\begin{itemize}

\item Fue posible lograr en general la detección de los tres tipos movimientos elegidos.

\item Se consiguió implementar con éxito un método poco utilizado en el ámbito del estudio de la dinámica de vehículos que determina un conjunto de características mínimas para discriminar cada evento.

\item Se determinó una combinación de las características encontradas para que por medio de una red neuronal se realizará la clasificación de los eventos.

\end{itemize} 

Como continuación al trabajo presentado se propone:
%\textcolor{blue}{Sin embargo, al ser esta una primera investigación acerca del tema, quedan aún muchas cosas por perfeccionar para lograr una mayor eficiencia en la detección.} \textcolor{red}{esto sería trabajo futuro no pongas que te falto mucho, sino que hay mucho por hacer}
\begin{itemize}
\item Incrementar la base de datos de los eventos para mejorar la detección.
\item Aumentar el número de características discriminantes.
\item Ajustar las regiones características de cada evento (por ejemplo, convertir los prismas las a elipsoides).
\end{itemize}

Se espera que el trabajo presentado pueda ser utilizado como una solución tangible en un futuro cercano.