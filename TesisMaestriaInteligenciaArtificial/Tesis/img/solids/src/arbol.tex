% !TEX encoding = UTF-8 Unicode

% NOTA: Este script hace uso de las imágenes previamente generadas: arreglo_arbol, cubo_[A-F] y prisma_H.

% Compilar con PDFLaTeX.
\documentclass[crop, rgb]{standalone}
\usepackage{tikz}
\usetikzlibrary{quotes, shadows}

\makeatletter
\def\input@path{
	{../definitions/}
	{../settings/}
	{../../../../def/}
}
\makeatother
% !TEX encoding = UTF-8 Unicode

% Parámetros principales
\usepackage{xcolor}

\definecolor{pagecolor}{HTML}{FFFFFF}
\definecolor{maintextcolor}{HTML}{1B1C1E}
%\definecolor{pagecolor}{HTML}{FFFFFF}
\definecolor{maintextcolor}{HTML}{1B1C1E}
%\definecolor{pagecolor}{HTML}{FFFFFF}
\definecolor{maintextcolor}{HTML}{1B1C1E}

\definecolor{rojo}{HTML}{C82121}	% DD3E3E	% 255, 0, 0
\definecolor{azul}{HTML}{0F4392}	% 11468F	% 0, 0, 169

\colorlet{colorLineaSolidos}{maintextcolor}
\colorlet{colorTextoSolidos}{colorLineaSolidos}

\newlength{\grosorLineaSolidos}
\setlength{\grosorLineaSolidos}{0.2pt}
\newlength{\grosorLineaSolidosPlanoDosD}
\setlength{\grosorLineaSolidosPlanoDosD}{1pt}

\def\solidesFont{New-Century-Schoolbook}

\colorlet{colorMalla}{gray!10}
\colorlet{colorLineasMalla}{maintextcolor}
\colorlet{colorTextoMalla}{colorLineasMalla}
\definecolor{colorCeldaObjeto}{gray}{0.2745}			% 70/255
\definecolor{colorCeldaVecina}{gray}{0.6274}			% 160/255
\definecolor{colorCeldaSegundaVecina}{gray}{0.8196}		% 209/255

\newlength{\grosorLineaMalla}
\setlength{\grosorLineaMalla}{2pt}
\newlength{\grosorLineaMallaArreglosDosD}
\setlength{\grosorLineaMallaArreglosDosD}{1pt}
\newlength{\grosorLineaMallaArreglosTresD}
\setlength{\grosorLineaMallaArreglosTresD}{1.6pt}

\colorlet{colorFlechas}{colorLineaSolidos}


\begin{document}
%
%
\pagestyle{empty}%
\nopagecolor%
%
\begin{tikzpicture}[every node/.style={inner sep=0, outer sep=0}]
\large
\def\heightCubeImg{1.05cm}
\def\heightPrismImg{1.2cm}
\def\ruta{../../}
\node at (-4, 2.5) {\includegraphics[width = 5cm]{\ruta arreglo_arbol}};
\begin{scope}[
	nodes = {
		draw, 
		circle, 
		minimum size = 1.5cm, 
		inner sep = -2pt, 
		line width = 1pt, 
		color = maintextcolor, 
		fill = colorMalla, 
		circular drop shadow
	}
]
\node (B) at (0,0) {\includegraphics[height=\heightPrismImg]{\ruta prisma_H}};
\node (C) at (2.5,0) {\includegraphics[height=\heightCubeImg]{\ruta cubo_C}};
\node (D) at (4.35,0) {\includegraphics[height=\heightCubeImg]{\ruta cubo_E}};
\node (E) at (6.75,0) {\includegraphics[height=\heightPrismImg]{\ruta prisma_H}};
\coordinate (J1) at (1.25,1.25);
\node (G) at (1.25,2.5) {\includegraphics[height=\heightCubeImg]{\ruta cubo_B}};
\node (H) at (4.35,2.5) {\includegraphics[height=\heightCubeImg]{\ruta cubo_D}};
\node (I) at (6.75,2.5) {\includegraphics[height=\heightCubeImg]{\ruta cubo_F}};
\coordinate (J3) at (2.8,3.75);
\node (J) at (3.7,5) {\includegraphics[height=\heightCubeImg]{\ruta cubo_A}};
\end{scope}

\begin{scope}
	[nodes = {inner sep = 1.7pt}, 
	line width = 1pt, 
	color = maintextcolor]
	\draw (B) edge ["$V$"] (J1);
	\draw (C) edge ["$V$", swap] (J1);
	\draw (J1) edge (G);
	\draw (D) edge ["$V$"] (H);
	\draw (E) edge ["$H$"] (I);
	\draw (G) edge ["$H$"] (J3);
	\draw (H) edge ["$H$", swap] (J3);
	\draw (J3) edge (J);
	\draw (I) edge ["$V$", swap] (J);
\end{scope}

\useasboundingbox 
	(current bounding box.south west) +(0pt, -7.8pt) 
	rectangle 
	(current bounding box.north east) +(11.6pt, 7.7pt);
\node[color = gray] at (1.15, 5.2) 
	{\textsf{Solución óptima}};
\draw
	[line width = 2pt, 
	color = gray, 
	dashed] 
	plot[smooth cycle, tension = 0.8] 
		coordinates{
			(7.6,3) 
			(7.75,1.25) 
			(7.75,-0.2) 
			(6.75,-1) 
			(5.75,0) 
			(5.65,2.5) 
			(3.5,4) 
			(2.7,5) 
			(3.7,6) 
			(5.2,5)
		};
\end{tikzpicture}
%
%
\end{document}%