\documentclass[letterpaper, 11pt]{memoir}
\usepackage[utf8]{inputenc}
\usepackage[default]{raleway}


% Configuraciones generales
\usepackage{geometry}
\geometry{
	right=15mm,
	left=15mm,
	top=25mm,
	bottom=20mm
}
\setlength{\parskip}{\baselineskip}
\setlength{\parindent}{0pt}
\pagestyle{empty}


% Fuentes
\usepackage[letterspace=200]{microtype}
\usepackage{fontawesome}
%\usepackage{fontspec}


% Varios
\usepackage{amsmath}
\usepackage{multicol}
\usepackage[table]{xcolor}
\usepackage{anyfontsize}
\usepackage{float}
\usepackage{calc}
\usepackage{xfp}
\usepackage{xargs}


% Pie de página para colocar fecha de compilación.
\usepackage{fancyhdr}
\fancyhf{}
\renewcommand{\headrulewidth}{0pt}

% Tablas
\usepackage{array}
\usepackage{supertabular}
\usepackage{cellspace}
\setlength{\tabcolsep}{0pt}
%\usepackage{longtable}
%\usepackage{tabu}

\newcounter{rowcount}
\setcounter{rowcount}{0}


% Gráficos
\usepackage{tikz}
\usepackage{pgfplots}
\tikzset{
    every picture/.style={%
        execute at begin picture={%
            \pgfsetbuttcap%
            \pgfsetmiterjoin%
            \pgfsetmiterlimit{10}%
            \pgfsetdash{}{0pt}%
        }
    }
}
\usetikzlibrary{fadings}

\definecolor{cafe}{rgb}{0.250980392,0,0}
\definecolor{verde}{rgb}{0,0.5,0}
\definecolor{azul}{rgb}{0,0,0.5}


% Títulos y secciones
\usepackage{titlesec}
\titlespacing*{\section}{0cm}{0.8cm}{-1.8\baselineskip}
\titleformat*{\section}{\lsstyle\color[rgb]{0.2,0,0}\bfseries\fontsize{12}{12}\selectfont}
\newlength{\sectionRulevSkip}
\setlength{\sectionRulevSkip}{2.5pt}


% Columnas por página
\setlength{\columnsep}{1cm}
\columnseprule 1pt
\def\columnseprulecolor{\color{gray!50}}


% Hipervínculos
\usepackage{hyperref}
\hypersetup{
	colorlinks=false,
	hidelinks=true,
}

\def\CIIAlocation{\href{https://maps.google.com/?cid=17992863989216810333}{\textcolor{red}{\faMapMarker}}}
\def\MIApage{\color{cyan!65}\href{https://www.uv.mx/dia/}{\faUniversity}}
\def\FFIApage{\color{cyan!65} \href{https://www.uv.mx/ffia/}{\faUniversity}}

\def\BlockchainCourseURL{\href{https://www.ude.my/UC-6573ff27-f962-4339-810e-ec8824785339/}{\Large\color{gray}\faShareSquareO}}
\def\TimeSeriesCourseURL{\href{https://ude.my/UC-51b1bb6c-b324-41d7-b11f-3826f7999574}{\Large\color{gray}\faShareSquareO}}
\def\GitCourseURL{\href{https://ude.my/UC-c48e4dd1-59c9-4f2c-b199-c9d09a74f9e8}{\Large\color{gray}\faShareSquareO}}
\def\SQLCourseURL{\href{https://ude.my/UC-6acc8434-7c34-40b5-ae1b-b813b2252e33}{\Large\color{gray}\faShareSquareO}}
\def\BasicSQLCourseURL{\href{https://www.sololearn.com/Certificate/1060-849693/pdf/}{\Large\color{gray}\faShareSquareO}}
\def\JavaCourseURL{\href{https://www.sololearn.com/Certificate/1068-849693/pdf/}{\Large\color{gray}\faShareSquareO}}
\def\DockerAndKubernetesCourseURL{\href{https://ude.my/UC-93d1fe9f-91d7-4547-9f2a-2d924a7582f3}{\Large\color{gray}\faShareSquareO}}
\def\MongoDBDataModelingCourseURL{\href{https://learn.mongodb.com/c/nsoZ-ybvTlyIW4GdYEmyDg}{\Large\color{gray}\faShareSquareO}}
\def\MongoDBAggregationFrameworkCourseURL{\href{https://learn.mongodb.com/c/izxrxl1BTveMe2NkrtiEtw}{\Large\color{gray}\faShareSquareO}}
\def\MongoDBPythonAPICourseURL{\href{https://learn.mongodb.com/c/1TWNDNs5RHKU1bY3O0K8GQ}{\Large\color{gray}\faShareSquareO}}
\def\MongoDBPerformanceCourseURL{\href{https://learn.mongodb.com/c/OII3PHyCQuSof4KOy6ahVA}{\Large\color{gray}\faShareSquareO}}
\def\MongoDBBasicClusterAdministrationCourseURL{\href{https://university.mongodb.com/course_completion/cd0248d4-64b2-4a81-9253-1766d7e3459f}{\Large\color{gray}\faShareSquareO}}
\def\MongoDBBasicsCourseURL{\href{https://university.mongodb.com/course_completion/93f517c6-c0f7-4ecc-8871-5524c33f1a48}{\Large\color{gray}\faShareSquareO}}


% Elementos con línea vertical y bullet.
\newlength{\sepVertical}
\setlength{\sepVertical}{26pt}

\newlength{\compExpPreBulletWidth}
\setlength{\compExpPreBulletWidth}{0.0\columnwidth}
\newlength{\compExpTimeLineWidth}
\setlength{\compExpTimeLineWidth}{0.2\columnwidth}
\newlength{\compExpTimeLineSep}
\setlength{\compExpTimeLineSep}{12.7pt}
\newlength{\compExpParWidth}
\setlength{\compExpParWidth}{0.61\columnwidth}

\newcommandx{\elementoInicio}[2][1=, usedefault]{%
	\noindent\hfill\parbox[t]{\columnwidth}{%
		\parbox[t]{\compExpPreBulletWidth}{\raggedleft #1}%
		{\hfill\LARGE$\color{cafe}\bullet$\hfill}%
		{\hspace{-\compExpTimeLineSep}\color{cafe}\vrule width 1pt height 5pt\hspace{\compExpTimeLineSep}}%
		\parbox[t]{\compExpParWidth}{#2}%
	}%
	\vspace{-1.1pt}%
}
\newcommandx{\elementoMitad}[2][1=, usedefault]{%
	\noindent\hfill\parbox[t]{\columnwidth}{%
		\parbox[t]{\compExpPreBulletWidth}{\raggedleft #1}%
		{\hfill\LARGE$\color{cafe}\bullet$\hfill}%
		{\hspace{-\compExpTimeLineSep}\color{cafe}\vrule width 1pt height \sepVertical\hspace{\compExpTimeLineSep}}%
		\parbox[t]{\compExpParWidth}{#2}%
	}%
	\vspace{-1.1pt}%
}
\newcommandx{\elementoFin}[2][1=, usedefault]{%
	\noindent\hfill\parbox[t]{\columnwidth}{%
		\parbox[t]{\compExpPreBulletWidth}{\raggedleft #1}%
		{\hfill\LARGE$\color{cafe}\bullet$\hfill}%
		{\hspace{-\compExpTimeLineSep}\color{cafe}\vrule width 1pt height \sepVertical depth -2pt\hspace{\compExpTimeLineSep}}%
		\parbox[t]{\compExpParWidth}{#2}%
	}%
}

\newlength{\timeLineWidth}
\setlength{\timeLineWidth}{1.5pt}
\newlength{\timeLineSep}
\setlength{\timeLineSep}{18.8pt}
\newlength{\companyCircleWidth}
\setlength{\companyCircleWidth}{1.5pt}
\newlength{\companyCircleRadious}
\setlength{\companyCircleRadious}{11.5pt}
\newlength{\companyCircleShift}
\setlength{\companyCircleShift}{-1.65\ht\strutbox}

\newlength{\vruleHeight}
\setlength{\vruleHeight}{-2\companyCircleRadious - \companyCircleShift}
\newlength{\vruleDepth}
\setlength{\vruleDepth}{-20pt -\companyCircleRadious - \companyCircleShift}
\newlength{\experienceItemOverlap}
\setlength{\experienceItemOverlap}{2.2pt}
\newlength{\experienceItemSep}
\setlength{\experienceItemSep}{0pt}
\newlength{\companyLogoDefaultHeight}
\setlength{\companyLogoDefaultHeight}{1.86\companyCircleRadious}

\newcommandx{\firstElement}[5][1=default.png, 2=\companyLogoDefaultHeight, 3={(0,0)}, 4=\experienceItemSep, usedefault]{%
	\noindent\parbox[t]{\columnwidth}{%
		\hfill%
		\begin{tikzpicture}[baseline={([yshift=\companyCircleShift]current bounding box.north)}]
			\useasboundingbox (-\companyCircleRadious, -\companyCircleRadious) rectangle (\companyCircleRadious, \companyCircleRadious);
			\draw[color=cafe, line width=\companyCircleWidth] (0, 0) circle (\companyCircleRadious);
			\node[shift={#3}] {\includegraphics[height=#2]{img/#1}};
		\end{tikzpicture}\hfill%
		{\hspace{-\timeLineSep}\color{cafe}\vrule width \timeLineWidth height \vruleHeight \hspace{\timeLineSep}}%
		\parbox[t]{0.85\columnwidth}{#5}%
	}%
	\vspace{#4 - \experienceItemOverlap}
}

\newcommandx{\middleElement}[5][1=default.png, 2=\companyLogoDefaultHeight, 3={(0,0)}, 4=\experienceItemSep, usedefault]{%
	\noindent\parbox[t]{\columnwidth}{%
		\hfill%
		\begin{tikzpicture}[baseline={([yshift=\companyCircleShift]current bounding box.north)}]
			\useasboundingbox (-\companyCircleRadious, -\companyCircleRadious) rectangle (\companyCircleRadious, \companyCircleRadious);
			\draw[color=cafe, line width=1.5pt] (0, 0) circle (\companyCircleRadious);
			\node[shift={#3}] {\includegraphics[height=#2]{img/#1}};
		\end{tikzpicture}\hfill%
		{\hspace{-\timeLineSep}\color{cafe}\vrule width \timeLineWidth height 20pt depth \vruleDepth \hspace{\timeLineSep}}%
		{\hspace{-\timeLineSep - \timeLineWidth}\color{cafe}\vrule width \timeLineWidth height \vruleHeight \hspace{\timeLineSep}}%
		\parbox[t]{0.85\columnwidth}{#5}%
	}%
	\vspace{#4 - \experienceItemOverlap}
}

\newcommandx{\lastElement}[5][1=default.png, 2=\companyLogoDefaultHeight, 3={(0,0)}, 4=\experienceItemSep, usedefault]{%
	\noindent\parbox[t]{\columnwidth}{%
		\hfill%
		\begin{tikzpicture}[baseline={([yshift=\companyCircleShift]current bounding box.north)}]
			\useasboundingbox (-\companyCircleRadious, -\companyCircleRadious) rectangle (\companyCircleRadious, \companyCircleRadious);
			\draw[color=cafe, line width=1.5pt] (0, 0) circle (\companyCircleRadious);
			\node[shift={#3}] {\includegraphics[height=#2]{img/#1}};
		\end{tikzpicture}\hfill%
		{\hspace{-\timeLineSep}\color{cafe}\vrule width \timeLineWidth height 20pt depth \vruleDepth \hspace{\timeLineSep}}%
		\parbox[t]{0.85\columnwidth}{#5}%
	}%
}


\renewcommand{\arraystretch}{1.56}
\newlength\barheight
\setlength{\barheight}{9.5pt}

\def\stepFunction#1{%
	\ifthenelse{\fpeval{#1 < 0} = 1}{0}{#1}%
}

\newlength{\totalBarLenght}%

\newcommandx{\skillBar}[4][2=\linewidth, 3=15, 4=\barheight, usedefault]{%
%
\setlength{\totalBarLenght}{#2}%
\def\shadingPercent{#3}%
\def\firstSegmentLenght{\fpeval{(#1 - \shadingPercent / 2)/100}\totalBarLenght}%
\def\shadingSegmentLenght{\fpeval{\shadingPercent / 100}\totalBarLenght}%
\def\secondSegmentLenght{\fpeval{(100 - #1 - \shadingPercent / 2)/100}\totalBarLenght}%
%
\def\auxFill{0.2pt}%
\def\firstPoint{\firstSegmentLenght}%
\def\secondPoint{\firstSegmentLenght + \shadingSegmentLenght}%
\def\thirdPoint{\firstSegmentLenght + \shadingSegmentLenght + \secondSegmentLenght}%
%
\begin{tikzpicture}%
	% The +1pt and -1pt in the clip is a workaraund for a shade overheight in some PDF viewers.
	\clip (0, 0 + 1pt) rectangle (\totalBarLenght, \barheight - 1pt);
	\fill[color=brown!300] (0, 0) rectangle (\firstPoint + \auxFill, \barheight);
	\shade[left color=brown!300, right color=gray!60] (\firstPoint, 0) rectangle (\secondPoint + \auxFill, \barheight);
    \fill[gray!60] (\secondPoint, 0) rectangle (\thirdPoint, \barheight); 
\end{tikzpicture}%
}


% Subrayado especial.
\usepackage{contour}
\usepackage[normalem]{ulem}
\renewcommand{\ULdepth}{1.8pt}
\contourlength{0.8pt}
\newcommand{\fancyUnderline}[1]{%
	\uline{\phantom{#1}}%
	\llap{\contour{white}{#1}}%
}


\begin{document}
%
%
%
% Encabezado.
\begin{tikzpicture}[remember picture, overlay]%
	\useasboundingbox (0cm,0cm) rectangle (\paperwidth,5cm);
	%
	\node[anchor=north west, minimum width=\paperwidth, minimum height=5cm, fill=brown!340, inner sep=0pt, align=center] at (current page.north west) {
		\\[35pt] \textcolor{white}{\fontsize{24}{24}\selectfont \textbf{JOSÉ ALBERTO} LÓPEZ LÓPEZ} \\[10pt] 
		\textcolor{white}{\large BACHELORS IN PHYSICS} \\[5pt] 
		\textcolor{white}{\large MASTERS IN ARTIFICIAL INTELLIGENCE} \\[5pt] 
		\textcolor{gray}{\href{mailto:JLL.6@hotmail.com}{\faEnvelope\ \ JLL.6@hotmail.com}}
	};
\end{tikzpicture}
%
\vspace{3.5\baselineskip}
%
%
\section*{ABOUT ME \vskip \sectionRulevSkip \hrule width 0.6\linewidth} 
%
%
Artificial Intelligence developer with experience in various languages and frameworks.
I have worked in big data projects using real time transmission technologies and several data science projects as well.
I consider myself a dedicated, responsable and very capable person.
%
\vspace{\baselineskip}
%
%
\begin{multicols*}{2}%
%
%
\section*{EXPERIENCE \vskip \sectionRulevSkip \hrule width 0.6\columnwidth}
%
%
\firstElement[Grupo_Salinas_logo.eps][1.2\companyCircleRadious][(-0.25pt, 2pt)]{
	\textbf{Artificial Intelligence Developer} \newline
	\textsc{Grupo Salinas} $\cdot$ Ciudad de México, Mexico \newline
	\color{black!70} dec 2021 -- aug 2022 \newline
	Big data AI developer in projects related to bank credit fraud prevention, using tools like Spark, Hadoop, Kafka, among others.
}
\middleElement[Migala.png]{
	\textbf{Data Scientist} \newline
	\textsc{Migala} $\cdot$ Ciudad de México, Mexico \newline
	\color{black!70} dec 2021 -- feb 2022 \newline
	Data analysis of an inquest realized to potential members of a new political party.
	
	Foundations of a Blockchain project with Solidity and other components.
}
\middleElement[Best_Practices_Consulting_logo.png][1.8\companyCircleRadious]{
	\textbf{Data Scientist} \newline
	\textsc{Best Practices Consulting} $\cdot$ Querétaro, Mexico \newline
	\color{black!70} mar 2021 -- sep 2021 \newline
	Data visualizations designer with Power BI for several companies projects, including a dashboard with information about gas distribution routes in real time.
}
\lastElement[UV.pdf][1.8\companyCircleRadious][(0, 0.15pt)]{
	\textbf{Algorithms and Prototipes Developer} \newline
	\textsc{Artificial Intelligence Research Center} $\cdot$ University of Veracruz, Mexico \newline
	\color{black!70} jun 2016 -- jun 2017 \newline
	C++ developer of an autonomous movement vehicle detection algorithm, designed for an embedded system, using accelerometers and Furier analisys.
}
%
\columnbreak%
%
%
\section*{PROGRAMMING\ \ LANGUAGES \vskip \sectionRulevSkip \hrule width \columnwidth}
%
%
\begin{tabular}{>{\bfseries\raggedleft} m{0.23\columnwidth} @{\hspace{8pt}}m{0.68\columnwidth}}%
	 & \textcolor{gray}{\textbf{Junior}} \hfill \textcolor{cafe}{\textbf{Senior}}\\
    Python & \skillBar{85} \\
    C++  & \skillBar{80} \\
    SQL & \skillBar{70} \\
    Java & \skillBar{50} \\
    Solidity & \skillBar{40} \\
    HTML, CSS & \skillBar{50} \\
    Javascript & \skillBar{30} \\
    Others & \skillBar{80} \\
\end{tabular}%
%
%
\vspace{-8pt}
\section*{FRAMEWORKS\ \ \&\ \ SYSTEMS\vskip \sectionRulevSkip \hrule width \columnwidth}
%
%
\begin{tabular}{>{\bfseries\raggedleft} m{0.23\columnwidth} @{\hspace{8pt}}m{0.68\columnwidth}}%
	Spark & \skillBar{80} \\
	MongoDB & \skillBar{75} \\
	Kafka & \skillBar{75} \\
	Hadoop & \skillBar{60} \\
	Git & \skillBar{50} \\
	Blockchain & \skillBar{50} \\
	Azure & \skillBar{40} \\
	Docker & \skillBar{40} \\
    \small Kubernetes & \skillBar{40} \\
\end{tabular}
%
\columnbreak
%
%
\section*{COMPLEMENTARY\ \ EXPERIENCE \vskip \sectionRulevSkip \hrule width \columnwidth}
%
%
\setlength{\sepVertical}{23pt}
\elementoInicio{Experience in Keras and TensorFlow for deep learning and machine learning.}
\elementoMitad{Knowledge of dense, convolutional, and recurrent neural networks; as well as evolutionary and nature-inspired algorithms.}
\elementoMitad{Computer vision with OpenCV library in C++ and Python.}
\elementoMitad{Experience in parallel programing with OpenMS and MSI libraries in C and C++.}
\elementoMitad{Experience working with data science applied to geographical and cities databases.}
\elementoMitad{Large experience in data visualization with Matplotlib, Gnuplot, Ti\textit{k}Z-PGF and Power BI.}
\elementoMitad{Mathematical foundations of models for time series analisys, like ARIMA, SARIMA, VARMA, Dickey-Fuller test, Granger causality, Holt-Winters method, EWMA, ETS, Hodrick-Prescott filter, etc.}
\elementoFin{Robotic Operative System (ROS) with C++ and Python.}
%
\vspace{-0.6\baselineskip}%
%
%
\section*{EDUCATION \vskip \sectionRulevSkip \hrule width 0.6\columnwidth}
%
%
{\renewcommand{\arraystretch}{1.7}%
\begin{tabular}{>{\centering}p{0.15\columnwidth} p{0.85\columnwidth}}
	2020 & %
	\textbf{Masters in artificial intelligence} \newline%
	\textsc{Artificial Intelligence Research Center} $\cdot$ %
	University of Veracruz, Mexico \ \MIApage \tabularnewline
	%
	2018 & %
	\textbf{Bachelors in physics} \newline%
	\textsc{Faculty of Physics} $\cdot$ %
	University of Veracruz, Mexico \ \FFIApage \newline%
	\color{black!70} Best marks of generation.
\end{tabular}}
%
\vspace{-0.6\baselineskip}%
%
%
\section*{LANGUAGES \vskip \sectionRulevSkip \hrule width 0.6\columnwidth}
%
%
\begin{tabular*}{\columnwidth}{>{\raggedright} p{0.185\columnwidth} >{\centering} p{0.05\columnwidth}}
    \textbf{Spanish} & Native
\end{tabular*}
%
\columnbreak
%
%
\section*{COURSES \vskip \sectionRulevSkip \hrule width 0.6\columnwidth}
%
%
{\renewcommand{\arraystretch}{1.9}%
\rowcolors{0}{}{gray!20}%
%
\centering%
\begin{tabular}{>{\centering\bfseries} m{0.33\columnwidth} >{\centering\bfseries} m{0.39\columnwidth} >{\centering\bfseries} >{\centering}m{0.28\columnwidth}}%
	\rowcolor{cafe}{\color{white} Institute} & {\color{white} Subject} & {\color{white} ID and URL} \tabularnewline
	Udemy & Docker \& Kubernetes & \DockerAndKubernetesCourseURL \tabularnewline
	MongoDB Academy & MongoDB & %
		\MongoDBBasicsCourseURL\ \ %
		\MongoDBBasicClusterAdministrationCourseURL\ \ %
		\MongoDBDataModelingCourseURL \linebreak
		\MongoDBAggregationFrameworkCourseURL\ \ %
		\MongoDBPythonAPICourseURL\ \ %
		\MongoDBPerformanceCourseURL \tabularnewline
	Udemy & Blockchain & \BlockchainCourseURL \tabularnewline
	Udemy & Time Series Data Analisis & \TimeSeriesCourseURL \tabularnewline
	Udemy & Git & \GitCourseURL \tabularnewline
	Udemy & SQL: Basic to Advanced & \SQLCourseURL \tabularnewline
	SoloLearn & SQL & \BasicSQLCourseURL \tabularnewline
	SoloLearn & Java & \JavaCourseURL \tabularnewline
\end{tabular}}
%
\vspace{\baselineskip}
%
%
\section*{PROJECTS \vskip \sectionRulevSkip \hrule width 0.6\columnwidth}
%
%
\textbf{Robot Hostess} \ {\color{red} \href{https://www.youtube.com/watch?v=KU4NL_bc_rA}{\faYoutubePlay}} \ {\href{https://github.com/JoseAlbertoLopezLopez/CV_projects/tree/master/RobotHostess}{\faGithub}}\newline
{\color{black!70} I programmed the entire code of a Robot hostess in C++ with ROS. The robot detects diners via face recognition and lead them to a suitable table.}

\textbf{Augmented reality} \ {\color{red} \href{https://www.youtube.com/watch?v=aLcEEVrU13I}{\faYoutubePlay}} \ {\href{https://github.com/JoseAlbertoLopezLopez/CV_projects/tree/master/AugmentedReality}{\faGithub}}\newline
{\color{black!70} Example of augmented reality made with Python and OpenCV.}

\textbf{Traveling salesman problem visualization} \ {\color{red} \href{https://www.youtube.com/watch?v=VK9NzBgPXBw}{\faYoutubePlay}} \ {\href{https://github.com/JoseAlbertoLopezLopez/CV_projects/tree/master/SimulatedAnnealing}{\faGithub}}\newline
{\color{black!70} Visualization of traveling salesman problem solved with simulated annealing.}

\textbf{Grasp Objects in Clutter (Masters Thesis)}\newline
{\color{black!70} I developed an algorithm for accommodate objects in order to facilitating later grasps by appling AI nature inspired algorithms.}
%
\columnbreak
%
%
\section*{CONTACT \vskip \sectionRulevSkip \hrule width 0.6\columnwidth}
%
%
\begin{itemize}
	\item[\fontsize{11}{11}\selectfont \href{mailto:JLL.6@hotmail.com}{\faEnvelopeO}] \href{mailto:JLL.6@hotmail.com}{JLL.6@hotmail.com}%
	\item[\fontsize{13}{13}\selectfont \faMobile \hspace{0.07cm}] +52 228-105-4832%
	\item[\fontsize{12}{12}\selectfont \color{azul} \faPhone] +52 296-964-7471%
	\item[\fontsize{12}{12}\selectfont \color{azul} \href{https://www.linkedin.com/in/alberto-l\%C3\%B3pez-177a281a0/?locale=en_US}{\faLinkedinSquare}] \href{https://www.linkedin.com/in/alberto-l\%C3\%B3pez-177a281a0/?locale=en_US}{\textcolor{blue}{\fancyUnderline{Alberto López}}}%
\end{itemize}
%
%
\section*{REFERENCES \vskip \sectionRulevSkip \hrule width 0.6\columnwidth}
%
%
\textbf{Marxlenin Zapata Yañez} \newline%
Chief of Reactor Engineering department at Laguna Verde Nuclear Power Plant. \newline%
\textcolor{black!70}{Relation: Previous boss.} \newline%
\href{mailto:marxlenin.zapata@cfe.gob.mx}{\faEnvelopeO\ marxlenin.zapata@cfe.gob.mx}

\textbf{Irving Morales Agiss, PhD} \newline%
Director of data science MORLANmx enterprise. \newline%
\textcolor{black!70}{Relation: Professor.} \newline%
\href{mailto:irvingfisica@gmail.com}{\faEnvelopeO\ irvingfisica@gmail.com}

\textbf{Antonio Marín Hernández, PhD} \newline%
Researcher at the Artificial Intelligence Research Center of University of Veracruz, Mexico. \newline%
\textcolor{black!70}{Relation: Thesis assesor.} \newline%
\href{mailto:anmarin@uv.mx}{\faEnvelopeO\ anmarin@uv.mx}
%
\thispagestyle{fancy}%
\rfoot{\textcolor{black!50}{\textit{Compilation date: \today}}}%
%
\end{multicols*}
%
%
%
\end{document}