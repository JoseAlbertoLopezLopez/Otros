% !TEX encoding = UTF-8 Unicode
\chapter{Conclusiones}\label{chap:conclusiones}
%
%
En el presente trabajo se diseñó un método para el acomodo de objetos en un espacio delimitado, el cual tiene como objetivo reducir el posterior costo de acceso a estos.
Para ello se abordó el problema de una forma simplificada de como se haría en una situación real.
Esto es, discretizando el espacio donde se colocan los objetos, así como reduciendo la complejidad geométrica de los objetos y la cantidad clases u objetos distintos que se pueden utilizar.
Debido a su naturaleza discreta, el problema se puede abordar como una búsqueda en un espacio de permutaciones, para lo cual existen diversas metodologías y algoritmos de búsqueda a elegir, de entre los cuales se optó por un algoritmo de recocido simulado.

Dado que no existe a la fecha una investigación que aborde el problema planteado, el aporte principal del presente trabajo es la definición de los parámetros y reglas para evaluar qué tan adecuado es un acomodo, de acuerdo a los fines establecidos.
Dicha evaluación está basada en la geometría de los objetos y en cómo esta afecta su sujeción cuando los objetos están en determinada configuración.
De forma que al combinar esto con el algoritmo de búsqueda elegido, se puede obtener el acomodo requerido.

Los resultados muestran que, en los casos en los que fue posible comparar el método propuesto con una búsqueda exhaustiva, se obtuvo en el 91.7\% de los casos, los mismos resultados que dicha búsqueda, esto es, un arreglo óptimo de acuerdo a las reglas establecidas.
Mientras que en casos más complejos, en los que no fue posible realizar una búsqueda exhaustiva, los resultados siguen siendo buenos, llegando a ser óptimos en varios casos en los que es posible determinar tal característica.
Con lo cual se puede aceptar la hipótesis de investigación propuesta, al menos para el problema definido.

Como trabajo futuro se propone expandir las capacidades del método desarrollado.
Esto es, adaptarlo para que se puedan abordar casos más complejos y diversos, por ejemplo, con una mayor cantidad de objetos, así como incluir más variedades de geometrías para estos, es decir, incluir una mayor cantidad de clases.
Además, se podrían relajar las restricciones acerca de cómo se pueden colocar los objetos en el espacio, permitiendo configuraciones en las que los objetos puedan ser colocados encima de otros y además puedan tener un número mayor de orientaciones.

El método de búsqueda también podría cambiar y mejorarse de acuerdo a estas expansiones.
Se podría implementar otra técnica de búsqueda, o bien, utilizar una combinación de estas para diseñar una metodología que se adapte de mejor manera al problema planteado.