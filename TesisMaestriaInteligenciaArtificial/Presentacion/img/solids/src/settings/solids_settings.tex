% !TEX encoding = UTF-8 Unicode

% Parámetros principales
\usepackage{xcolor}

\definecolor{pagecolor}{HTML}{FFFFFF}
\definecolor{maintextcolor}{HTML}{1B1C1E}
%\definecolor{pagecolor}{HTML}{FFFFFF}
\definecolor{maintextcolor}{HTML}{1B1C1E}
%\definecolor{pagecolor}{HTML}{FFFFFF}
\definecolor{maintextcolor}{HTML}{1B1C1E}

\definecolor{rojo}{HTML}{C82121}	% DD3E3E	% 255, 0, 0
\definecolor{azul}{HTML}{0F4392}	% 11468F	% 0, 0, 169

\colorlet{colorLineaSolidos}{maintextcolor}
\colorlet{colorTextoSolidos}{colorLineaSolidos}

\newlength{\grosorLineaSolidos}
\setlength{\grosorLineaSolidos}{0.2pt}
\newlength{\grosorLineaSolidosPlanoDosD}
\setlength{\grosorLineaSolidosPlanoDosD}{1pt}

\def\solidesFont{New-Century-Schoolbook}

\colorlet{colorMalla}{gray!10}
\colorlet{colorLineasMalla}{maintextcolor}
\colorlet{colorTextoMalla}{colorLineasMalla}
\definecolor{colorCeldaObjeto}{gray}{0.2745}			% 70/255
\definecolor{colorCeldaVecina}{gray}{0.6274}			% 160/255
\definecolor{colorCeldaSegundaVecina}{gray}{0.8196}		% 209/255

\newlength{\grosorLineaMalla}
\setlength{\grosorLineaMalla}{2pt}
\newlength{\grosorLineaMallaArreglosDosD}
\setlength{\grosorLineaMallaArreglosDosD}{1pt}
\newlength{\grosorLineaMallaArreglosTresD}
\setlength{\grosorLineaMallaArreglosTresD}{1.6pt}

\colorlet{colorFlechas}{colorLineaSolidos}