\makeatletter


% algorithm2e

% Cambios para definir el indentado de líneas subsecuentes de comentarios igual a la 
% longitud ocupada por el símbolo de comentario y eliminar el espacio antes de este.
\renewcommand{\SetKwComment}[3]{%
  \algocf@newcommand{#1}{\@ifstar{\csname algocf@#1@star\endcsname}{\csname algocf@#1\endcsname}}%
	\algocf@newcommand{algocf@#1}[1]{%
      \ifthenelse{\boolean{algocf@hangingcomment}}{\relax}{\algocf@seteveryparhanging{\relax}}%
      \sbox\algocf@inputbox{\CommentSty{\hbox{#2}}}%
      \ifthenelse{\boolean{algocf@commentsnumbered}}{\relax}{\algocf@seteveryparnl{\relax}}%
      {\renewcommand{\algocf@endmarkcomment}{#3}%
        \let\\\algocf@endstartcomment%
        \algocf@startcomment\CommentSty{%
          \strut\ignorespaces##1\strut\algocf@fillcomment#3}\par}%
      \algocf@linesnumbered% reset the numbering of the lines
      \ifthenelse{\boolean{algocf@hangingcomment}}{\relax}{\algocf@reseteveryparhanging}%
    }%
  %%% side comment definitions
	\algocf@newcommand{algocf@#1@star}[2][]{%
      \ifArgumentEmpty{##1}\relax{% TODO: Is this even necessary, with all those \ifx's?
        \ifthenelse{\boolean{algocf@scleft}}{\setboolean{algocf@sidecomment}{true}}{\setboolean{algocf@sidecomment}{false}}%
        \ifx##1h\setboolean{algocf@altsidecomment}{true}\SetSideCommentLeft\fi%
        \ifx##1f\setboolean{algocf@altsidecomment}{true}\SetSideCommentRight\fi%
        \ifx##1l\setboolean{algocf@altsidecomment}{false}\SetSideCommentLeft\fi%
        \ifx##1r\setboolean{algocf@altsidecomment}{false}\SetSideCommentRight\fi%
      }%
      \ifthenelse{\boolean{algocf@hangingcomment}}{\everypar{\algocf@everyparnl\hangafter=1\hangindent=\CommentHangIndent\relax}}{\relax}%
      \sbox\algocf@inputbox{\CommentSty{\hbox{#2}}}%
      \ifthenelse{\boolean{algocf@commentsnumbered}}{\relax}{%
      	\renewcommand{\algocf@everyparnl}{\relax}%
      	\everypar{\algocf@everyparnl\hangafter=1\hangindent=\CommentHangIndent\relax}}%
      {%
        \renewcommand{\algocf@endmarkcomment}{#3}%
        \let\\\algocf@endstartsidecomment%
        % here is the comment
        \ifthenelse{\boolean{algocf@altsidecomment}}{\relax}{\@endalgocfline}%
        \algocf@scrfill\algocf@startsidecomment\CommentSty{%
          \strut\ignorespaces##2\strut\algocf@sclfill#3}\algocf@scpar%
      }%
      \algocf@linesnumbered% reset the numbering of the lines
      \ifArgumentEmpty{##1}\relax{%
        \ifthenelse{\boolean{algocf@sidecomment}}{\setboolean{algocf@scleft}{true}}{\setboolean{algocf@scleft}{false}}%
        \setboolean{algocf@altsidecomment}{false}%
      }%
	}%
  }%

% Redefinición del caption.
\renewcommand{\algocf@captionproctext}[2]{%
	\addtolength{\hsize}{1.5\algomargin}%
	\begin{minipage}{\linewidth + 1.5\algomargin}%
	{%
		\ProcSty{\ProcFnt\algocf@procname%
			\ifthenelse{\boolean{algocf@procnumbered}}%
			{\nobreakspace\thealgocf\algocf@typo\algocf@capseparator}%
			{\relax}%
		}%
		\nobreakspace\ProcNameSty{\ProcNameFnt\algocf@captname\slshape #2@}%
		\ifthenelse{\equal{\algocf@captparam #2@}{\arg@e}}{}{%
			\ProcNameSty{$\ProcNameFnt($}\ProcArgSty{\ProcArgFnt\algocf@captparam #2@}\ProcNameSty{$\ProcNameFnt)$}% 
		}%
		\algocf@captother #2@%
	}%
	\end{minipage}
}%


% footmisc

% Márgen izquierdo de nota al pie.
\def\@makefntext#1{%
	\ifFN@hangfoot
		\bgroup
		\setbox\@tempboxa\hbox{%
			\ifdim\footnotemargin>0pt
				\hb@xt@\footnotemargin{\@makefnmark\hss}%
			\else
				\@makefnmark
			\fi
		}%
		\setlength{\leftmargin}{\wd\@tempboxa + 1em}%	% Línea modificada.
		\rightmargin\z@
		\linewidth \columnwidth
		\advance \linewidth -\leftmargin
		\parshape \@ne \leftmargin \linewidth
		\@totalleftmargin \leftmargin
		\footnotesize
		\@setpar{{\@@par}}%
		\leavevmode
		\llap{\box\@tempboxa}%
		\parskip\hangfootparskip\relax
		\parindent\hangfootparindent\relax
	\else
		\parindent1em
		\noindent
		\ifdim\footnotemargin>\z@
			\hb@xt@ \footnotemargin{\hss\@makefnmark}%
		\else
			\ifdim\footnotemargin=\z@
				\llap{\@makefnmark}%
			\else
				\llap{\hb@xt@ -\footnotemargin{\@makefnmark\hss}}%
			\fi
		\fi
	\fi
	\footnotelayout#1%
	\ifFN@hangfoot
		\par\egroup
	\fi
}


\makeatother