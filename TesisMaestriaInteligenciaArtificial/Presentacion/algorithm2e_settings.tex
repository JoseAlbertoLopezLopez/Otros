\usepackage[linesnumbered, nokwfunc, lined, titlenotnumbered, hanginginout]{algorithm2e}

\SetKwInput{Data}{Datos de entrada}
\SetKwInput{Result}{Resultado}
\SetKwInput{Definitions}{Definiciones}
\SetKw{Not}{not}
\SetKw{Or}{or}
\SetKwIF
	{IfWithNoThen}
	{ElseIfWithNoThen}
	{Else}
	{if}
	{}
	{else if}
	{else}
	{end}
\DontPrintSemicolon
\SetAlgoSkip{}
\IncMargin{1.1em}
\SetNlSkip{0.4em}
\SetInd{0.3em}{0.8em}
\RestyleAlgo{titlenotnumbered}
\SetAlgorithmName{Algoritmo}{}{}
\SetAlgoFuncName{Funci\'on:}{}
\SetAlCapFnt{\hspace{5pt}\footnotesize}
\SetProcNameFnt{\footnotesize\itshape}
\SetProcArgFnt{\footnotesize}
\SetProcFnt{\footnotesize}

\def\f#1{\textsl{#1}}

\newcommand{\QuarterBlankLine}{\vskip 0.25ex}
\newcommand{\HalfBlankLine}{\vskip 0.5ex}
\newcommand{\HalfQuarterBlankLine}{\vskip 0.75ex}
\newcommand{\OneQuarterBlankLine}{\BlankLine\vskip 0.25ex}
\newcommand{\OneHalfBlankLine}{\BlankLine\vskip 0.5ex}
\newcommand{\TwoBlankLines}{\BlankLine\BlankLine}

% Estilo de comentarios.
\SetKwComment{tcp}{\!\fontsize{9}{9}\faComments[regular]\:}{}
\newcommand\commentstyle[1]{\ttfamily\textcolor{gray!80!black}{\textit{\footnotesize #1}}}
\SetCommentSty{commentstyle}

% Comandos para modificar números de líneas.
\let\oldnl\nl
\newcommand{\nonl}{\renewcommand{\nl}{\let\nl\oldnl}\stepcounter{AlgoLine}}
% Sin aumentar contador.
\let\oldnl\nl
\newcommand{\noln}{\renewcommand{\nl}{\let\nl\oldnl}}

\makeatletter
\setboolean{algocf@hangingcomment}{true}
\let\oldalgocf@captother\algocf@captother
\renewcommand\algocf@captother{\footnotesize\oldalgocf@captother}	

% Margen izquierdo al caption.
\renewcommand{\algocf@captionproctext}[2]{%
  \hspace{5pt}\begin{minipage}{\linewidth-5pt}%
  {%
    \ProcSty{\ProcFnt\algocf@procname\ifthenelse{\boolean{algocf@procnumbered}}{\nobreakspace\thealgocf\algocf@typo\algocf@capseparator}{\relax}}%
    \nobreakspace\ProcNameSty{\ProcNameFnt\algocf@captname #2@}% Name of the procedure in ProcName Style. 
    \ifthenelse{\equal{\algocf@captparam #2@}{\arg@e}}{}{% if no argument, write nothing
      \ProcNameSty{\ProcNameFnt(}\ProcArgSty{\ProcArgFnt\algocf@captparam #2@}\ProcNameSty{\ProcNameFnt)}%else put arguments in ProcArgSty:
    }% endif
   {\setstretch{0}\algocf@captother #2@\par}%		Cambiar interlineado.
  }%
  \end{minipage}
}%

\makeatother

\SetAlCapNameFnt{\footnotesize}
\renewcommand{\thealgocf}{}
\newcounter{function}
\makeatletter
\AtBeginEnvironment{function}{\let\c@algocf\c@function}
\makeatother
\SetKwProg{Function}{function}{:}{end}

\SetKwProg{FunctionInicio}{function}{:}{}
\SetKwProg{FunctionFin}{}{}{end}
\SetKwIF{IfInicio}{ElseIf}{Else}{if}{then}{else if}{else}{}
\SetKwIF{IfFin}{ElseIf}{Else}{}{}{else if}{else}{end}